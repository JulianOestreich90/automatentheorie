\documentclass[11pt]{beamer}
\usetheme{CambridgeUS}
\usepackage[utf8]{inputenc}
\usepackage[german]{babel}
\usepackage{amsmath}
\usepackage{amsfonts}
\usepackage{amssymb}
\usepackage{nameref}
\usepackage{mathtools}
\usepackage{graphicx}
\usepackage{tikz}
\usetikzlibrary{shapes}
\author{Julian Oestreich}
\title[Minimierung von Baumautomaten]{Satz von Myhill-Nerode für reguläre Baumsprachen und Minimierung von Baumautomaten}
%\setbeamercovered{transparent} 
\setbeamercolor{block title}{bg=red!30, fg=black}
\setbeamertemplate{navigation symbols}{}

\newif\ifplacelogo % create a new conditional
\placelogotrue % set it to true

\logo{\ifplacelogo\includegraphics[height=1.2cm]{uni_logo.pdf}\fi} 
\institute[Universität Leipzig]{\small{¸Seminarmodul Automatentheorie} \\ Universität Leipzig} 
%\date{} 
%\subject{} 


\makeatletter
\newcommand*{\currentsection}{\@currentlabelname}
\newcommand*{\theheader}{\thesection.\thesubsection.  \currentsection}
\newcommand{\superhyperref}[2]{
\color{blue}
\href{#1}{#2}
\color{black}}
\makeatother


\begin{document}


\begin{frame}
\titlepage
\end{frame}

\placelogofalse

\section{Erinnerung: Äquivalenzrelationen}
\begin{frame}{\thesection. \currentsection}
\begin{itemize}
	\item Eine Relation \mbox{$ \equiv\subseteq A \times A $} nenn man \textit{Äquivalenzrelation} gdw. sie \textit{reflexiv, transitiv} und \textit{symetrisch} ist¸
	\item Die Menge \mbox{$[a]_\equiv \coloneqq \{ a^{'} |  a^{'} \equiv a\} $} wird \textit{Äquivalenzklasse} von \textit{a} genannt
	\item Eine Äquivalenzrelation hat \textit{endlichen Index}, wenn sie endlich viele Äquivalenzklassen bildet
\end{itemize}
\end{frame}

\section{Kongruenz}
\begin{frame}{\thesection. \currentsection}

	\begin{block}{Kongruenz} Eine Äquivalenzrelation \mbox{$\equiv$}¸ auf \mbox{$\mathcal{T(F)} $} ist eine \textit{Kongruenz} gdw. 
	$$\forall f \in \mathcal{F}_n:(\forall i \leq n: u_i \equiv v_i) \Longrightarrow f(u_1,...,u_n) \equiv f(v_1,...,v_n) $$
	\end{block}
	\begin{itemize}
	\item Intuition: Funktionen sind äquivalent bei Anwendung auf äquivalente Funktionsargumente
	\item $\equiv$ ist kongruent gdw. abgeschlossen unter (1-löchrigen) Kontexten, d.h. $$ \forall C u v : u \equiv v \Longrightarrow C[u] \equiv C[v]$$
	\end{itemize}
\end{frame}

\begin{frame}{\thesection. \currentsection}
Für eine Sprache L definieren wir die Kongruenz $\equiv_L$ als
$$ u \equiv_L v\; gdw.\; \forall C: C[u] \in L\; gdw.\; C[v] \in L $$
\begin{itemize}
	\item Offensichtlich \textit{Äquivalenzrelation}
	\item Offensichtlich \textit{kongruent}
	\item Intuition: Die Sprache L unterscheidet nicht zwischen \textit{u} und \textit{v}
\end{itemize}
\end{frame}

\section{Deterministische endliche Baumautomaten}
\begin{frame}{\thesection. \currentsection}
Vollständige DFTA $(Q,\mathcal{F},Q_f,\delta)$
\begin{itemize}
	\item mit $\delta : (\mathcal{F}_n \times Q^{n} \rightarrow Q)_n $
	\item entspricht $\Delta$:
	$$ f(q_1,...,q_n) \rightarrow q \; gdw. \; \delta(f, q_1,...,q_n) = q  $$
	\item Erweiterung für Bäume:
	$$ \delta(f(t_1,...,t_n)) = \delta(f, \delta(t_1),...,\delta(t_n))$$
	\item kompatibel mit $\rightarrow_A$, d.h
	$$ t \rightarrow_A q \; gdw. \; \delta(t) = q $$
\end{itemize}
\end{frame}


\section{Das Myhill-Nerode Theorem}
\begin{frame}{\thesection. \currentsection}
\begin{block}{Anwendung}
	\begin{itemize}
		\item \textit{Notwendiges und hinreichendes Kriterium} dafür ob $\mathcal{L}$ \textit{regulär} ist.
		\item Verglichen mit \textit{Pumping-Lemma} kann Myhill-Nerode immer Aussage tätigen ob $\mathcal{L}$ regulär.
		\item Anzahl der Zustände des¸ DFTA $\mathcal{A}_{min}$ der $\mathcal{L}$ akzeptiert entspricht dem Index der Äquivalenzklassen der Nerode Relation $\sim$
	\end{itemize}
\end{block}¸
\end{frame}

\begin{frame}{\thesection. \currentsection}
\begin{block}{Theorem}
	Diese 3 Aussagen sind äquivalent:
	\begin{enumerate}
		\item L ist eine erkennbare Baumsprache
		\item L ist die Vereinigung einiger Äquivalenzklassen von endlicher Kongruenz
		\item Die Relation \mbox{$ \equiv_L $} ist eine endliche Kongruenz
	\end{enumerate}
\end{block}
\end{frame}

\subsection{Beweis 1 $\rightarrow$ 2}
\begin{frame}{\thesection.\thesubsection. \currentsection}
\begin{itemize}
	\item L reg. $\rightarrow$ L wird von DFTA $(Q,\mathcal{F},Q_f,\delta)$ erkannt
	\item $\delta$ Übergangsfunktion
	\item Ann.: $\equiv_A$ auf T($ \mathcal{F} $): $u \equiv_A v \; gdw. \; \delta(u)=\delta(v)$
	\item offens. kongruent
	\item $\equiv_A¸$ hat endlichen Index (max. $|Q|$)
	\item Also haben wir $L = \cup \{[u] \; | \; \delta(u) \in Q_f\} $
\end{itemize}
\end{frame}

\subsection{Beweis 2 $\rightarrow$ 3}
\begin{frame}{\thesection.\thesubsection. \currentsection}
\begin{itemize}
	\item $\sim$ Kongruenz mit endl. Index
	
	\item $\mathcal{F(T)}\//\sim \; = \{\underbrace{[t_1],[t_2],...,[t_i]}_{\substack{L\;{\overset{(2)}{=}} \bigcup_{j=1}^{i} [t_j] }},...,[t_n]\}$
	\item Ann.: u $\sim$ v
	\item Dann \begin{align*}
	\forall C &\in C(\mathcal{F}): \; C[u] \sim C[v] \\
	&{\overset{(i)}{\longrightarrow}}C[u] \in L  \\
	&{\overset{(ii)}{\longrightarrow}} \; \exists j: C[u] \in t_j \\
	&{\overset{(i)}{\longrightarrow}} C[v] \in t_j \\
	&{\overset{(ii)}{\longrightarrow}} C[v] \in L
	\end{align*}
\end{itemize}
\end{frame}

\begin{frame}{\thesection.\thesubsection. \currentsection}
\begin{itemize}
	\item $ C[v] \in L \longrightarrow C[u] \in L$ analog
	\item $\Longrightarrow \forall C: C[u] \in L \Longleftrightarrow C[v] \in L$ 
	\item $u \equiv_L v $
	\item da $ (u,v) \in \;¸ \sim \; \longrightarrow (u,v) \in \equiv_L$
	\item und $\sim \; \subseteq \; ¸\equiv_L$, da $\sim$ \textit{Verfeinerung} von $\equiv_L$
\end{itemize}
¸\end{frame}

\subsection{Beweis 3 $\rightarrow$ 1}
\begin{frame}{\thesection.\thesubsection. \currentsection}
\begin{itemize}
	\item 
\end{itemize}
\end{frame}



\end{document}
